% !TEX encoding = UTF-8 Unicode
\documentclass[11pt, oneside]{article}
\usepackage{geometry}
\geometry{letterpaper}                   		% ... or a4paper or a5paper or ... 	
\usepackage{amssymb}
\usepackage[utf8]{inputenc}

%SetFonts

%SetFonts

\newtheorem{theoreme}{Théorème}
\newtheorem{proposition}{Proposition}

\title{Reformulation d'Abbot-Hanson}
\author{Richard Abou Chaaya, Thomas Bianco, Benoît Ferté, Gabriel Merlin, Mario Michelessa \\ \\
Encadrants : Joanna Tomasik, Arpad Rimmel}
\date{}

\begin{document}
\maketitle

Ce document propose une reformulation du théorème 2.1 ci-dessous de \cite{AbbotHanson} dans le cas particulier des nombres de Schur,
en partant de deux partitions sans-sommes quelconques.

\begin{theoreme}
Soient $p$ et $q$ deux entiers naturels; $S(p + q) \geqslant 2 S(p) S(q) + S(p) + S(q).$
\end{theoreme}

\section{Construction de la partition}

Pour prouver ce résultat, nous allons partir de deux partitions sans-sommes quelconques $(\mathrm{P}_i)_{1 \leqslant i \leqslant p}$
et $(\mathrm{Q}_j)_{1 \leqslant j \leqslant q}$, respectivement à $p$ et $q$ ensembles.
Notons $m$ et $n$ les entiers tels que $\{1, \dots, m\} = \mathrm{P}_1 \cup \dots \cup \mathrm{P}_p$
et $\{1, \dots, n\} = \mathrm{Q}_1 \cup \dots \cup \mathrm{Q}_q$.
Nous allons construire une partition sans-somme de $\{1, \dots, nN + m\}$ à $p+q$ ensembles à partir de ces deux partitions,
où $N \geqslant m + 1$ est un entier, que nous devrons prendre dans $\{2m, 2m+1\}$ pour que cela puisse marcher.

Comme dans Abbot-Hanson, définissons deux ensembles $\mathrm{A}$ et $\mathrm{B}$~:
\begin{equation}
 \mathrm{A} = \{ bN + c, 0 \leqslant b \leqslant n, 1 \leqslant c \leqslant m \}
\end{equation}
et
\begin{equation}
 \mathrm{B} = \{ bN - c, 1 \leqslant b \leqslant n, 0 \leqslant c \leqslant N-m-1 \}.
\end{equation}
Lorque l'on pose $c' = N - c$ et $b' = b-1$ dans la définition de $\mathrm{B}$, celle-ci se réécrit
\[ \mathrm{B} = \{ b'N + c', 0 \leqslant b' \leqslant n-1, m+1 \leqslant c' \leqslant N \}. \]
Cette réécriture souligne le fait que l'intervalle $\{1, \dots, nN + m \}$ est partitionné selon les restes pour la division euclidienne par $N$~:
si $x \in \{1, \dots, nN + m \}$ a un reste compris entre $1$ et $m$, alors $x \in \mathrm{A}$;
sinon, si $x$ a un reste appartenant à $\{0\} \cup \{m+1, \dots, N-1\}$, alors $x \in \mathrm{B}$.

Construisons maintenant une partition sans-somme $(\mathrm{R}_k)$ de $\{1, \dots, nN + m \}$ à $p+q$ ensembles en suivant Abbot-Hanson.
Quelque soit $i \in \{1, \dots, p\}$, soit $\mathrm{R}_i$ l'ensemble des éléments $bN + c$ de $\mathrm{A}$ tels que $c \in \mathrm{P}_i$,
et quelque soit $j \in \{1, \dots, q\}$, soit $\mathrm{R}_{p + j}$ l'ensemble des éléments $bN - c$ de $\mathrm{B}$ tels que $b \in \mathrm{Q}_j$.
Par construction, $(\mathrm{R}_k)_{1 \leqslant k \leqslant p+q}$ constitue une partition de $\{1, \dots, nN + m \}$.

Trouvons d'abord une condition sur $N$ pour que les ensembles $\mathrm{R}_i$, $1 \leqslant i \leqslant p$, soient sans-sommes.
Si jamais il existait trois éléments $b_1 N + c_1$, $b_2 N + c_2$, $b_3 N + c_3$ appartenant à un même $\mathrm{R}_i$, tels que
\begin{equation}
 (b_1 N + c_1) + (b_2 N + c_2) = b_3 N + c_3,
\end{equation}
alors
\begin{equation}
 c_3 - c_2 - c_1 = (b_1 + b_2 - b_3)N,
\end{equation}
donc $N$ diviserait $c_3 - c_2 - c_1$. Par ailleurs,
\begin{equation}
 -(2m - 1) \leqslant c_3 - c_2 - c_1 \leqslant m-2,
\end{equation}
donc si l'on impose $N > 2m - 1$, la seule manière que $N$ divise $c_3 - c_2 - c_1$ est que $c_3 - c_2 - c_1 = 0$,
contredisant le caractère sans-somme de $\mathrm{P}_i$.

Trouvons ensuite une nouvelle condition sur $N$ contraignant les $\mathrm{R}_{p + j}$, $1 \leqslant j \leqslant q$ à être sans-sommes.
Supposons à nouveau qu'il puisse exister trois éléments $b_1 N - c_1$, $b_2 N - c_2$, $b_3 N - c_3$ appartenant à un $\mathrm{R}_{p + j}$
tels que $(b_1 N - c_1) + (b_2 N - c_2) = b_3 N - c_3,$ d'où
\begin{equation}
 c_3 - c_2 - c_1 = (b_3 - b_2 - b_1)N.
\end{equation}
Comme
\begin{equation}
 -2(N -m -1) \leqslant c_3 - c_2 - c_1 \leqslant N -m -1,
\end{equation}
imposer $N > 2(N -m -1)$ garantirait $c_3 - c_2 - c_1 = 0$,
entraînant aussi $b_3 - b_2 - b_1 = 0$ ce qui serait absurde.
Ainsi, la condition $N > 2(N -m -1)$, qui équivaut à $N < 2m + 2$, force les $\mathrm{R}_{p + j}$ à être sans-sommes.

Quand les deux conditions sont regroupées, il vient finalement $N \in \{2m, 2m+1\}$.

Pour obtenir l'inégalité annoncée, il suffit de prendre $m = S(p)$ et $n = S(q)$~:
une partition sans-somme de $\{1, \dots, S(p)(2S(q)+1) + S(q) \}$ à $p+q$ éléments ayant été construite, $S(p+q) \geqslant S(p)(2S(q)+1) + S(q)$.

\section{Propriétés de la partition}

Regardons les entiers susceptibles d'être ajoutés à une partition construite selon la méthode d'Abbot-Hanson.
Etant donné un entier $a$ et $\mathrm{E}$ une partie de $\mathbf{N}$, notons $v_a(\mathrm{E})$
le nombre de couples d'entiers $(a', a'') \in \mathrm{E}\times\mathrm{E}$ tels que $a' + a'' = a$.

Considérons à nouveau deux partitions sans-sommes quelconques $(\mathrm{P}_i)_{1 \leqslant i \leqslant p}$
et $(\mathrm{Q}_j)_{1 \leqslant j \leqslant q}$, et reprenons les mêmes notations que précédemment.
Soit $a$ un entier compris entre $nN + m + 1$ et $2nN - 1$.
Il peut s'écrire sous la forme $a = bN + c$, où $n \leqslant b \leqslant 2n-1$ et $0 \leqslant c \leqslant N-1$.

Si un couple $(c', c'') \in \mathrm{P}_i \times \mathrm{P}_i$ a pour somme $c$,
alors les $2n-b+1$ couples $(b' N + c', (b-b')N + c'')$, $b-n \leqslant b' \leqslant n$, appartiennent à $\mathrm{R}_i \times \mathrm{R}_i$
et vérifient $(b'N + c') + ((b-b')N + c'') = bN + c = a$, donc $v_a(\mathrm{R}_i) \geqslant (2n-b+1) v_c(\mathrm{P}_i)$.

Réciproquement, si $(a', a'') \in \mathrm{R}_i \times \mathrm{R}_i$ a pour somme $a$,
en écrivant $a' = b' N + c'$ et $a'' = b'' N + c''$, $0 \leqslant b', b'' \leqslant n$ et $(c', c'') \in \mathrm{P}_i \times \mathrm{P}_i$,
\[ bN + c = (b' + b'')N + (c'+c'') ,\]
et le fait que $N > 2m-1$ impose $c = c' + c''$ et $b' + b'' = b$.
En outre, $0 \leqslant b' \leqslant n$, d'où l'inégalité inverse $v_a(\mathrm{R}_i) \leqslant (2n-b+1) v_c(\mathrm{P}_i)$.

Faisons de même pour les ensembles $\mathrm{R}_{p+j}$.

Si $(b', b'') \in \mathrm{Q}_j \times \mathrm{Q}_j$ a pour somme $b$,
alors les $2(N-m-1)-c+1$ couples $(b' N + c', b''N + c - c')$, $c-N+m-1 \leqslant c' \leqslant N-m-1$, appartiennent à $\mathrm{R}_{p+j} \times \mathrm{R}_{p+j}$
et vérifient $(b'N + c') + (b''N + c-c') = bN + c = a$.

Réciproquement, si $(a', a'') \in \mathrm{R}_{p+j} \times \mathrm{R}_{p+j}$ a pour somme $a$,
en écrivant $a' = b' N + c'$ et $a'' = b'' N + c''$, $0 \leqslant c', c'' \leqslant N-m-1$ et $(b', b'') \in \mathrm{Q}_j \times \mathrm{Q}_j$,
\[ bN + c = (b' + b'')N + (c'+c'') ,\]
et le fait que $N < 2m$ impose $c = c' + c''$ et $b' + b'' = b$.

Ainsi, $v_a(\mathrm{R}_{p + j}) = (2N-2m-c-1) v_b(\mathrm{Q}_j)$.

\begin{proposition}
Si il n'est pas possible de rajouter l'entier $m+1$ à la partition sans-somme $(\mathrm{P}_i)_{1 \leqslant i \leqslant p}$
ni $n+1$ à $(\mathrm{Q}_j)_{1 \leqslant j \leqslant q}$ en conservant le caractère sans-somme,
alors $nN + m + 1$ ne peut être rajouté $(\mathrm{R}_k)_{1 \leqslant k \leqslant p+q}$.
\end{proposition}

Par hypothèse, quel que soit $1 \leqslant i \leqslant p$, il existe $c'$ et $c''$ dans $\mathrm{P}_i$ tels que $c' + c'' = m+1$,
donc $nN + m + 1 = (nN + c') + c''$, avec $nN + c' \in \mathrm{R}_i$.

De même, quel que soit $1 \leqslant j \leqslant q$, il existe $b'$ et $b''$ dans $\mathrm{Q}_j$ tels que $b' + b'' = n+1$,
donc $nN + m + 1 = (n+1)N + m - N + 1 = (b'N - (N - m - 1)) + b''N$, avec $b'N - (N - m - 1)$ et $b''N \in \mathrm{R}_{p + j}$.

\begin{thebibliography}{1}
 \bibitem{AbbotHanson}
  H. L. Abbot, D. Hanson, \textit{A problem of Schur and its generalizations}, Acta Arithmetica, XX (1972), pp. 175-187
\end{thebibliography}

\end{document}
