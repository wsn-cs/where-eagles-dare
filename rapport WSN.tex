\documentclass{report}
\usepackage[utf8]{inputenc}
\usepackage{hyperref}
\usepackage{geometry} \geometry{hmargin=1.5cm,vmargin=1.5cm}

\title{Weak Schur Numbers, 1�re version}
\author{Beno�t Fert�, Gabriel Merlin, Thomas Bianco, Mario Michelessa, Richard Abou Chaaya}
\date{Avril 2019}



\begin{document}
\renewcommand{\thesection}{\arabic{section}} 



\maketitle

\tableofcontents
\newpage

\section{Description}
On appelle nombre de Schur, not� $S(k)$, le plus grand entier N tel que :
\begin{center}
    
$ \exists A_1... A_k \ tels \ que \ \bigcup(A_i) = \{1,N\} $

et

$\forall i\in \{1,k\} \not\exists \ a,b,c\in A_i \ tels \  que \ a+b=c  $

\end{center}
Dans le cas o� $a \neq b$, on appellera N nombre de Schur faible, not� WS(k).
\\
On connait actuellement les nombres de Schur jusqu'� 5, et les nombres de Schur faibles jusqu'� 4. Pour les valeurs sup�rieures, on ne connait que des bornes, sup�rieures et inf�rieures. L'enjeu de ce projet est de parvenir � am�liorer ces bornes.
\\


\begin{tabular}{c|c|c|c|c|c|c|c|c|c|c|c|c}
    \hline
    k & 1 & 2 & 3 & 4 & 5 & 6 & 7 & 8 & 9 & 10 & 11 & 12 \\ \hline
    S(k) & 1 & 4 & 13 & 44 & 160 [6] &  &  &  &  &  &  & \\
    borne inf &  &  &  &  &  & 536 [2] & 1680 [2] & 5041 [5] & 15124 [5] & 51120 [7] & 172216 [7] & 575664 [7] \\ 
    borne sup &  &  &  &  &  & 1927 & 13490 & 107921 & 971290 & 8956901 & 98525912 & 1182310945 \\ \hline
    WS(k) & 2 & 8 & 23 & 66 &  &  &  &  &  &  &  &  \\ 
    borne inf &  &  &  &  & 196 [1] & 582 [3] & 1740 [4] & 5201 [4] & 15596 [4] & 51520 & 172216 & 575664 \\
    borne sup &  &  &  &  & 1630 & 11742 & 95900 & 876808 & 8877690 & 98641010 & 1193556233 & 15624736140 \\ \hline
    
    Valeur max & 2 & 8 & 23 & 66 & 196 & 582 & 1740 & 5201 & 15596 & 44455 [9] & 127575 [9] & 372389 [9] \\ 
    atteinte &  &  &  &  &  &  &  &  &  &  &  &  \\ \hline
    
\end{tabular}

Les bornes sup�rieures viennent de [9] et [10]

\section{Planning}

Divers articles seront �tudi�s tout au long du projet.
\\ En parall�le, des codes seront impl�ment�s pour tenter de trouver des bornes les plus hautes possibles.
\\ L'objectif � court terme est de r�ussir � implanter l'algorithme de Nested Monte-Carlo 

\section{Avancement}

Articles lus (et compris) : 
\begin{itemize}
    \item An abstract procedure to compute weak Schur numbers, B.Bouzy
    \item Lower bounds for S(n), Abbott et Hanson
\end{itemize}
Codes r�alis�s : 
\begin{itemize}
    \item Monte-Carlo na�f
    \item nested Monte-Carlo (� debugger)
\end{itemize}

\section{Annexes}
Des slides, supports pour une pr�sentation orale, ont �t� r�alis�es � chaque s�ance.
\\ Les diff�rents codes se trouvent sur un git.

\begin{thebibliography}{}
\bibitem{1}
 S. Eliahou, J. M. Marin, M. P. Revuelta and M. I. Sanz. Weak Schur numbers and the search for G. W. Walker�s lost partitions, in Computers & Mathematics with Applications, Vol. 63, 2012, 175�182.
\bibitem{2}
 H. Fredricksen and M. M. Sweet. Symmetric sum-free partitions and lower bounds for Schur numbers, in Electronic Journal of Combinatorics, Vol. 7, R32, 2000.
\bibitem{3}
  S. Eliahou, C. Fonlupt, J. Fromentin, V. Marion-Poty, D. Robilliard and F. Teytaud. Investigating Monte-Carlo methods on the weak Schur problem, in Evolutionary Computation in Combinatorial Optimization, Vol. 7832 of Lecture Notes in Computer Science, Springer, Berlin, 2013, 191�201.
\bibitem{4}
 F. Rafilipojoana. LOWER BOUNDS ON THE WEAK SCHUR NUMBERS UP TO 9 COLORS.
\bibitem{5}   
  I. Schur. Uber die Kongruenz xm + ym = zm (mod p), in Jahresbericht der Deutschen Mathematiker-Vereinigung, Vol. 25, 1917, 114�116
\bibitem{6}
 M. Heule. Schur Number Five, in the Thirty-Second AAAI Conference on Artificial Intelligence 
\bibitem{7}
 Abbott and Hanson, A problem of Schur and its generalizations, Acta Arithmetica XX
\bibitem{8}
 A lower bound for weak Schur numbers with a deterministic algorithm, Ghada Ben Hassine, Pierre Berg�, Arpad Rimmel, Joanna Tomasik
\bibitem{9}
 Irving, 1973, An extension of Schur�s theorem on sum-free partition
\bibitem{10}
 P. Bornsztein. On an extension of a theorem of Schur.Acta Arith., 101:395�399, 2002
\end{thebibliography}



\end{document}
